\documentclass[a4paper, 12pt]{report}

%%%%%%%%%%%%
% Packages %
%%%%%%%%%%%%

\usepackage[english]{babel}
\usepackage[noheader]{packages/sleek}
\usepackage{packages/sleek-title}
\usepackage{packages/sleek-theorems}
\usepackage{packages/sleek-listings}

%%%%%%%%%%%%%%
% Title-page %
%%%%%%%%%%%%%%

%\logo{./resources/pdf/logo.pdf}
%\institute{Game Design Document}

%\department{Department of Anything but Psychology}
\title{Game Title}
\author{\textit{Author}\\First name \textsc{Last name}}
%\supervisor{Linus \textsc{Torvalds}}
%\context{Well, I was bored...}
\date{\today}

%%%%%%%%%%%%%%%%
% Bibliography %
%%%%%%%%%%%%%%%%

%%%%%%%%%%
% Others %
%%%%%%%%%%

\lstdefinestyle{latex}{
    language=TeX,
    style=default,
    %%%%%
    commentstyle=\ForestGreen,
    keywordstyle=\TrueBlue,
    stringstyle=\VeronicaPurple,
    emphstyle=\TrueBlue,
    %%%%%
    emph={LaTeX, usepackage, textit, textbf, textsc}
}

\FrameTBStyle{latex}

\def\tbs{\textbackslash}

%%%%%%%%%%%%
% Document %
%%%%%%%%%%%%

\begin{document}
    \maketitle
    
    \tableofcontents

\chapter{Overview}

\section{Game Analysis}
Introduce the game. Present information on why this game will be fun, the purpose of the game, what the player does, and so on. This is meant to be a quick analysis of the game and what you can expect from it. Shouldn’t be more than 1-2 paragraphs.

\section{Mission Statement}
In 1-2 sentences, explain the game as if you were pitching it to potential players. This should be very intriguing. It typically includes the title, genre, platform, and brief idea of what the player does or has to overcome.


\section{Genre}
\begin{itemize}
    \item List or describe the game’s genre/genres
\end{itemize}
    

\section{Platforms}
\begin{itemize}
    \item List or describe the platforms the game will be made for.
\end{itemize}

    

\section{Target Audience}
Provide information on the audience the game is targeted to. Add details and information on the intended audience such as their habits, behaviors, likes, and dislikes. Are you targeting your game to a specific age group or perhaps people that enjoy certain genres? Is your intended audience from specific communities or will their locale play a role?

\chapter{Story-line \& Characters}
This is where you present a story synopsis, and discuss how the story will unfold as the player makes his or her way through the game. Include information on the key characters in the game, including descriptions, abilities, characteristics, how they fit into the story, how they affect gameplay, what the player will learn from them, etc. 

\begin{tabular}[h]{p{4 cm}|p{4 cm}|p{4cm}}
    \hline
    Character & Description & Characteristics
     \\
        \hline
         Character name & Describe the character. It is a playable character or NPC. How does this character fit into the story, etc. & Describe the character’s abilities, personality and so forth.
     \\
     \hline
\end{tabular}
\vspace{1 cm}

\chapter{Gameplay}

\section{Overview of Gameplay}
Include information on the game genre and how it is different, similar, or a hybrid of existing genres. Discuss what platform the game will be on, if it is going to be on multiple platforms discuss ways the game will be modified for each platform. Also, provide a general overview of the game modes available in single player and multiplayer. Also, list the Key Gameplay Features (selling features) of the game.

\section{Player Experience}
Provide a general overview of how the player experiences the game. Walk them through the screens they will see, what the level looks like and what their character can do. Give them a brief idea of objectives \& hazards they will face.  This should be in a second-person point of view using the word “you” to tell a story to the audience (players).

\section{Gameplay Guidelines}
This is a set of guidelines that the game must adhere to throughout the development process. These include rules for what is allowed and not allowed in the game. For instance, if you are creating a game for children, you will want to define guidelines for the level of violence presented in the game, what language can be used, and so on. 

\section{Game Objectives \& Rewards}
This is where you present more details on how the gameplay will motivate the player to progress through the game. Discuss rewards and penalties and the difficulty level. You can use the table below to help break down objectives and rewards.

\begin{tabular}[h]{p{4 cm}|p{4 cm}|p{4cm}}
    \hline
    Rewards & Penalties & Difficulty levels
     \\
        \hline
         List ways of how the player is rewarded and when. & Discuss things that hinder the player on progressing & Discuss the difficulty levels within the game
     \\
     \hline
\end{tabular}
\vspace{1 cm}

\section{Gameplay Mechanics}
This is where you start getting more specific on how some of the systems in the game will work. This includes how characters move in the game, what gameplay actions are available, item inventory and attributes, and how the game progresses from level to level.

\textbf{Character Attributes}
\begin{itemize}
    \item Character name: List the characters abilities \& how the player can perform them
\end{itemize}

\textbf{Game mode}
\begin{itemize}
    \item Game mode: Describe the objectives, hazards in the game mode. And discuss how the player progresses from level to level
\end{itemize}

\textbf{Scoring System}
\begin{itemize}
    \item Scoring attribute (coin, stars, xp): Describe how the player obtains this and the benefits. For instance, does getting more points unlock a special level.
\end{itemize}

\section{Level Design}

Discuss the levels. How many levels will the game have, what will be included in each level. Include overall look and feel, hazards the level presents, difficulty, objectives, etc.

\textit{Each level should be represented as sub-section}

\subsection{Level name}
\textit{picture}

\begin{itemize}
    \item List or describe how the level is accessed or unlocked.
\end{itemize}

\begin{itemize}
    \item List or describe the level’s look, difficulty, hazards, and objectives.
\end{itemize}

\chapter{Control scheme}
Describe the control setup for the game. Does your game use touch input, a controller, or mouse \& keyboard? Discuss the functionality of each button/touch. It may help to insert a diagram/pic to help explain the actions.

\begin{tabular}[h]{p{4 cm}p{7cm}}
    \hline
    Button & Action it performs
     \\
        \hline
         List the button & Describe what functionality the button press has within the game.
     \\
     \hline
\end{tabular}
\vspace{1 cm}

\section{Game Aesthetics \& User Interface}
Discuss the design techniques to be used. Describe the look \& shape of the characters, environment and pathways. Will the game look realistic or have some other art style. Discuss what type of theme the game will have \& what type of emotional impact you are hoping players experience. Discuss how the player’s gestures/interactivity has an affect on the visual experience. 

Present a general overview of the UI. How will the buttons be laid out, how will the HUD work, how does the menu system function, and so on. It is a good idea to insert photos, diagrams or concept art to help explain the UI.


\end{document}
